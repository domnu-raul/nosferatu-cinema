\documentclass[a4paper]{article}
\usepackage[a4paper, margin=.85in]{geometry}
\usepackage[none]{hyphenat}
\usepackage[utf8]{inputenc}
\usepackage{hyperref}

\hypersetup{
    pdftitle={Nosferatu Cinema - Web Design Project Documentation},
}

\title{Nosferatu Cinema\\Web Site Documentation}
\author{Raul Istrat\\
Computer Science Departament at\\
West University of Timisoara\\
\href{mailto:raul.istrat03@e-uvt.ro}{\texttt{raul.istrat03@e-uvt.ro}}}
\date{}

\begin{document}
\maketitle

\begin{abstract}
    Nosferatu Cinema is not a cinema, but I build the web site for it as a project for Web Design. On the "home" page, the viewer can see all movies that are screening in the following 7 days. On the "movies" page, viewers can see all screening dates and times, users may also select a movie to only display the screenings of that movie. The user may create an account and log in, although this doesn't offer any extra functionality for the website, but the "admin" user will get access to the "dashboard" page, where they can add or delete movies or screenings from the database.
\end{abstract}

\section{Introduction}




\end{document}